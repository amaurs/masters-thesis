This chapter elaborates on how the model was trained and compared with other techniques from classical computer vision. 

\section{Exploratory analysis}

Drone images from threed different towns in the state of Oaxaca where obtained from the National Center for Disaster Prevention (CENAPRED). This pictures where taken during the week following an earthquake that originated in the Pacific coast of the state. We have 727 images from Santa Maria Xadani, 1872 from Union Hidalgo, and 1134 images from Juchitan of Zaragoza. 


A t-sne analysis was performed with the images, and it is shown in the figure. To this end, the information from the pixels of each images was flattened into a vector comprising the means and standard deviations. This was a simple dimensionality reduction technique and was useful to embed the images into a lower dimensional space. As we can see, images form natural clusters depending on the town that they where taken from. This was somewhat expected because of the light conditions during the exposure tend to have less variance among similar times and places.

\begin{figure}[h]
  \begin{center}
    \subfigure{\label{fig:hausdorff}\fbox{\includegraphics[width=.75\textwidth]{src/t-sne.png}}}
  \end{center}
\end{figure}

The results obtained after applying t-sne, supports the proposed methodology of using images from one town, and try to predict on others. Due to the lack of training data, this was needed to be generated from the sample data. The application detailed in the previous chapter was used to crop and classify $100$ square patches from the images in each of the towns. Each patch was $327\times 327$ and a tag was assigined manualy by the author in each of them.

\section{Model validation}

In order to have an effective algorithm we need that it can perform well in places with images it has never seen before. To test this hipothesis we tried two different methodologies based on cross validation. First of all, classic n-fold cross validation was perform on the pool of training data generated for this experiment. This is, the training set was divided in $n$ subsets and then a model was trained using $n-1$ of those sets while the remaining one was used for testing the model. This was repeated n times leaving a different set each time for testing.

\section{Threshold selection}

The binary classifier assigns a real number in the interval $[0,1]$. To decide which values will be assigined with either level a threshold must be chosen. This was picked using a ROC curve. The ROC curve helps us chosing the performance that fits our needs in the ver possible way. For instance

\section{How much is enough}

In this section we want to create a benchmark on how much images are needed to perform a retraining 

\section{Computer vision versus convolutional neural networks}

\section{Results}
