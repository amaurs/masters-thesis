Los terremotos son un fenómeno impredecible cuyo nivel de destrucción puede llegar a ser catastrófico. Debido a su escasez, la correcta canalización de los recursos es vital en el proceso de mitigación de las secuelas. La colaboración masiva demostró ser un recurso fundamental para afrontar las secuelas que provocó el sismo del 19 de septiembre de 2017 en la Ciudad de México. El uso de las redes sociales permitió la colaboración entre las labores de rescate, logística y la sociedad civil. Sin embargo, ¿qué pasa cuando no existen las condiciones ni la infraestructura tecnológica propias de las grandes urbes? En este trabajo se exploran el uso de tecnología de punta en estos casos. Tomando como objeto de estudio imágenes capturadas por drones durante los días posteriores al sismo del 7 de septiembre de 2017, se propone un proceso de análisis que permite detectar zonas dañadas de manera automatizada. Para lograrlo, se utilizan redes neuronales convolucionales para reconocer edificios dañados en las imágenes. Los resultados indican que es posible generar un mapa de daños usando este proceso. Dicho mapa podría ser de gran utilidad en las labores de evaluación de daño para minimizar costos y facilitar su operación.