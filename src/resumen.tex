Los terremotos son un fen\'omeno impredecible cuyo nivel de destrucci\'on puede llegar a ser catastr\'ofico. Debido a su inevitable naturaleza, buscar alternativas que permitan una respuesta r\'apida y efectiva es una tarea de alta prioridad. La correcta canalizaci\'on de los recursos es vital en el proceso de mitigaci\'on de las secuelas.\\

La colaboraci\'on masiva demostr\'o ser un recurso fundamental para afrontar el sismo del 19 de septiembre de 2017 en la Ciudad de M\'exico. El uso de las redes sociales permiti\'o la colaboraci\'on entre las labores de rescate, log\'istica y la sociedad civil. Sin embargo, ?`qu\'e ocurre cuando no existen las condiciones ni la infraestructura tecnol\'ogica propias de las grandes urbes? En este trabajo se propone un m\'etodo que contempla estos casos.\\

Durante los d\'ias posteriores al sismo del 7 de septiembre de 2017, drones del Centro Nacional para la Prevenci\'on de Desastres (CENAPRED) inspeccionaron las zonas afectadas sobre tres pueblos en el estado de Oaxaca. Las im\'agenes obtenidas fueron liberadas y sirven como objeto de estudio para nuestra investigaci\'on.\\

Partiendo de una red neuronal convolucional entrenada con un banco de im\'agenes dise\~nado para la investigaci\'on en el campo de reconocimiento de objetos, se ajust\'o un modelo que detecta edificios da\~nados en las fotos aereas. Con ayuda de t\'ecnicas de georectificaci\'on, el modelo predictivo se us\'o para generar un mapa de da\~nos potenciales. Los resultados sugieren que este proceso podr\'ia servir para generar herramientas que ayuden a la correcta asignaci\'on de los recusos para minimizar costos y facilitar su operaci\'on.