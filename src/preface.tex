It is a very exiting moment for the field of computer vision, machines are now capable of performing task that we thought were impossible reaching new milestones each year. It has been a long time since computers were just huge furniture in cold university rooms. Computer power has grown exponentially since those days. Lately techiniques that where discarded because the were computation intense are now being unburried and have been showing great results in today machines.\\

I wanted to explore the possibilities that these techniques can offer in classic problems such as landcover classification. In particualar, I wanted to teach a neural network how to recognize a specific element in aereal imagery and then use the tailored features as input to train classic classifiers such as random forest or support vector machines.\\

This work explores the use of Convolutional Neural Networks in the context of a classical remote sensing problem called landcover classification.\\

This preface serves as an introduction to this work. It gives a short review on the contents of each chapter, and shows how is the thesis structured.\\

Motivations are explored in chapter 1. We expose why our work is important and we give a clear explanation about the objective of the experiment. Additionally, we mention the scope of the project.\\

An extensive literature review is shown in chapter 2. All the way back to the famous tecnique to analyse handwritten digits with the convolutional architecture that started the revolution. A review of more modern applications of the techique in more complex situations such as object recognition in images. We explore other experiments that show different uses of the CNNs in remote sensing. Damage assesment in the aftermath of natural disasters is also explored as it was the main motivation for this study.\\

Chapter 3 unveils the mathematical details that make this technique to work. The training process using backpropagation is explained. Last layer activations functions such as the sigmoid and softmax are inspected. A brief summary of a convulutional neural network techniques such as the ReLU activation, pooling, and data augmentation is given.\\

The architecture of our pipeline is explained in chapter 4. This includes the data gathering, data curation, the training of the network and the prediction. Details of the data base are given, including the reasoning behind some decisions. This chapter also explores how did we obtained the images.\\

The results of the experiment are exposed in chapter 5. Accuracy of the classifier, as well as the validation protocol are clearly exposed here. We also show the final classification map and talk about how it was obtained.\\

Finally, in chapter 6, we talk about future work, and what are the main conclusions that this work led us to. We include several improvements that can be made to the process in order to obtain better results.\\

This work was the result of an intership spent on the Stevens Insitute of Technology in Hoboken, New Jersey, during the summer of 2017. It was done under the supervision of Andrea García Tapia and José Emmanuel Ramirez Marquez.\\